\vspace{10pt}
\section{Analysis of Design Constraints}\label{sec:wr}
\subsection{Overview}
As shown in Figure.~\ref{fig:modeling}, in order to write or read the
cross point array, the external voltages should be applied at the end of
the word line and the bit line. Although there are several potential
read/write schemes can be used to program the memory array, it is
difficult to point out which schemes is the most proper choice under given
design constraints of area/energy/reliability. Therefore, in this section,
studies on different operation schemes and present are conducted. The
results of this study can be very useful to guide the design of the cross
point array. Since it is impossible to consider all of the data pattern
stored in the array, in this work, the best and worst cases are studied.

Table~\ref{table:parameter} shows the circuit parameter of our baseline
design. The data is consistent to the recently published studies on
ReRAM~\cite{crossbar_TED_2010}\cite{memristor:Cong}. In this section, the
reliability, energy consumption and area overhead for the four write
schemes are detailed. Then the sensitivities of these schemes to the data
pattern of HRS and LRS ReRAM cells, and interconnect wires are studied.

\begin{table}[!b]
  \centering
  \scriptsize
    \scriptsize
  \caption{Parameters of the baseline Cross Point Array}\label{table:parameter}
  \vspace{-5pt}
%  \begin{tabular}{|cccccp{3.5cm}|}
  \begin{tabular}{c|c|c}
    \hline    \hline
    % after \\: \hline or \cline{col1-col2} \cline{col3-col4} ...
    \textbf{Metric} & \textbf{Description} & \textbf{Values} \\
    \hline
    \textbf{$S_{cell}$} & Cell Size & \textbf{$4F^2$} \\
    \textbf{$R_l$} &  Interconnection Resistance&\textbf{$1.25\Omega$} \\
    \textbf{$R_s$} &  Resistance of SA&\textbf{$100\Omega$} \\
    \textbf{$V_{RESET}$} & Threshold voltage for RESET&\textbf{$2.0V$} \\
    \textbf{$V_{SET}$} & Threshold voltage for SET&\textbf{$-2.0V$} \\
    \textbf{$V_{READ}$} & Read Voltage of Cell&\textbf{$0.5V$} \\
    \textbf{$R_{off}$} & HRS Resistance &\textbf{$500K\Omega$} \\
    \textbf{$R_{on}$} & LRS Resistance &\textbf{$10K\Omega$} \\
    \textbf{$V_{W}(R)$} & Word Line Voltage during Read &\textbf{$\pm 1V ???$} \\
    \textbf{$V_{W}(W)$} & Word Line Voltage during Write  &\textbf{$0 / 2V$} \\
    \textbf{$V_{W}(H)$} & Half Selected Word Line Voltage &\textbf{$1V$} \\
    \textbf{$V_{B}(R)$} & Bit Line Voltage during Read  &\textbf{$10K\Omega ????$} \\
    \textbf{$V_{B}(W)$} & Bit Line Voltage during Write  &\textbf{$0 / 2V$} \\
    \textbf{$V_{B}(H)$} & Half Selected Bit Line Voltage &\textbf{$1V$} \\
    \textbf{$M$} & Number of Word Line &\textbf{$64$} \\
    \textbf{$N$} & Number of Bit Line &\textbf{$64$} \\

    \hline
  \end{tabular}
  \vspace{-10pt}
\end{table}

Considering that program schemes for write and read operation are
different and the the requirement for write and read are also dissimilar,
in the following section we carefully study the write and read operation
separately. And then the results are combined together to provide a design
methodology for the cross point array.

\subsection{Write Operation}
To write a ReRAM cell, a external voltage is required to applied across the cell for a certain duration. Intuitively, there are four possible schemes for the write operation:
\begin{enumerate}
  \item According the location of the target cell, activate one word line and one bit line and leave all of the other lines floating (FWFB shemes).
  \item Activate the targeted word line and bit line. Left all the other word line floating and half bias the other bit line (FWHB shemes).
  \item In contrast with the scheme 2, activate the targeted word line and bit line. Left all the other bit line floating and half bias the other wold line (HWFB shemes).
  \item Activate the targeted word line and bit line. Half bias all of the other bit line (HWHB shemes).
\end{enumerate}
Since the reliability, energy consumption and area overhead for these
schemes are different from each other. We will address these problem
separately and then combine all of the constraints to provide a design
guideline for write operation.

\vspace{10pt} \textbf{Reliable Write Operation.} \vspace{8pt}

The most important issue for the write operation is the reliability
concern. In the ideal condition, the resistances of interconnection wires
and the sneak currents at unselected cells are negligible. Therefore, all
of these four schemes can provide enough voltage drop across the specified
cell. However, the realistic circuit is not perfect and the electronic
behavior of the array will deviate from the ideal scenario with different
data pattern stored. A reliable write operation can be defined as:
switching the selected cells into required states without disturbing the
states of unselected cells. Therefore, there exist two potential write
error: \textbf{write failure} and \textbf{write disturbance}. All of the
write schemes should at least meet the reliability requirement at the
worst case. On the other word, the designer should make sure there is not
any write failure and write failure disturbance occur even in the worst
case.

First of all, we will discuss the inherent problem of FWFB scheme, which
may result in severs write disturbance. A worse case scenario for FWFB
write disturbance can be defined as: all of unselected cells in the
activated word line (or all of unselected cells in the activated bit line)
are at HRS and other cells are in LRS. In this case, the voltage drop at
unselected cells are mainly applied at the HRS cell at the word line (or
bit line). Figure.~\ref{fig:FWFR} shows a example of this case at a $64
\times 64$ cross point array. It clearly that all of the unselected cells
at the activated bit line will be disturbed. This inherent problem exist
at all of the FWFB schemes and become very serious with the large On-OFF
resistance ratio. Considering that the reported On-OFF resistance ratio of
ReRAM cell is always $>50$
~\cite{ReRAM_IEDM2010_Ho,ReRAM_IEDM2010_Chien,ReRAM_IEDM2010_Lee_Diode,ReRAM_IEDM2010_Lee_Evidence,ReRAM_ISSCC2011_Sheu,ReRAM_ISSCC2011_Otsuka},
it is impossible to build a cross point structure ReRAM with the FWFB
scheme. Therefore, in the following discussion, we only compare the
results of FWHB, HWFB and HWHB schemes.

\begin{figure}%[!hb]
\centering
  % Requires \usepackage{graphicx}
  \includegraphics[width=0.45\textwidth]{./figures/FWFB2.pdf}\\
  \caption{Write Disturbance for FWFB Schemes. ( $V_{W32} = 2V$, $V_{B32} = 0V$. $R_{x,32}$ at HRS, others at LRS.) }\label{fig:FWFR}
\end{figure}

The write failure results from the voltage drop along the word line and
bit line. It has been shown that the worst case of this scenario is:

In order to make sure

\begin{figure}%[!hb]
\centering
  % Requires \usepackage{graphicx}
  \includegraphics[width=0.45\textwidth]{./figures/worst_v.pdf}\\
  \caption{Write Voltage Requirement. }\label{fig:worst_v}
\end{figure}




\begin{figure}%[!hb]
\centering
  % Requires \usepackage{graphicx}
  \includegraphics[width=0.45\textwidth]{./figures/shape.pdf}\\
  \caption{Write Voltage Requirement with Different Memory Shape. }\label{fig:shape}
\end{figure}

However, with the increase of the driven voltage, the voltage drop across
the unselected cell will also increase. Therefore, a write disturbance
will occur when the voltage applied at the half select cell exceeds the
threshold voltage for SET and RESET operation. Figure.~\ref{fig:half}
shows the worst case (maximum) voltage applied at the unselected cell.





1. Only consider the one-side scheme
Consider 1/2 1/3 and floating?
Energy Issue? Define Energy Efficient Parameter?
Reliable Issue: Any possibility of write disturbance




2. Present the voltage drop problem
Reliability Issue

3. Two Way Scheme

\subsection{Read Operation}
