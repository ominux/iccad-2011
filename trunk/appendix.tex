\label{sec:app}

As mentioned in Section~\label{sec:model}, the model is built upon KCL,
therefore, the current equations for each cross-point can be set following
\begin{equation}\label{equ:KCL0}
 {\Sigma}_{I=1}^kI_k=0.
\end{equation}
Besides, for the sake of brevity, we assume that the wordline voltage
drivers only locate at the edge of $V_{W1} \sim V_{Wm}$ and bitline
multiplexes locate at the edge of $V_{B1} \sim V_{Bn}$. Points locate at
the other two edges are left floating.

First of all, for normal points, which located inside the memory array,
the KCL equations take the form of
\begin{equation}\label{equ:KCL1}
R_l^{-1}V_{i,j-1} -(2R_l^{-1}+R_{i,j}^{-1})V_{i,j}+ R_l^{-1}V_{i,j+1}+R_{i,j}^{-1}V'_{i,j}=0,
\end{equation}
for the node at wordline layer and
\begin{equation}\label{equ:KCL2}
R_l^{-1}V'_{i-1,j} -(2R_l^{-1}+R_{i,j}^{-1})V'_{i,j}+ R_l^{-1}V'_{i+1,j}+R_{i,j}^{-1}V_{i,j}=0,
\end{equation}
for all of the nodes with $1<i<m$ and $1<j<n$ in a $m \times n$ array.

For all of the points $V_{i,1}$ ($1\leq i\leq m$), according to different
write scheme, they can be connected to the voltage driver $V_{Wi}$ (as
activated points) or left floating (as floating points). For the activated
points, we have
\begin{equation}\label{equ:KCL3}
 -(R_v^{-1}+R_l^{-1}+R_{i,1}^{-1})V_{i,1}+ R_l^{-1}V_{i,2}+R_{i,1}^{-1}V'_{i,1}=-R_v^{-1}V_{Wi},
\end{equation}
and for floating points, we have
\begin{equation}\label{equ:KCL4}
 -(R_l^{-1}+R_{i,1}^{-1})V_{i,1}+ R_l^{-1}V_{i,2}+R_{i,j}^{-1}V'_{i,1}=0.
\end{equation}

Similarly, for the points of $V'_{1,j}$ ($1\leq j\leq n$), the KCL
equations take the form of
\begin{equation}\label{equ:KCL5}
 -(R_s^{-1}+R_l^{-1}+R_{1,j}^{-1})V'_{1,j}+ R_l^{-1}V'_{2,j}+R_{1,j}^{-1}V_{1,j}=-R_s^{-1}V_{Bj},
\end{equation}
for activated points and
\begin{equation}\label{equ:KCL6}
 -(R_l^{-1}+R_{1,j}^{-1})V'_{1,j}+ R_l^{-1}V'_{2,j}+R_{1,j}^{-1}V_{1,j}=0.
\end{equation}
for floating points.

Finally, all of the other points at $V_{i,n}$ and $V'_{m,j}$ ($1\leq i\leq
m, 1\leq j\leq n$) are floating points and have the form of
\begin{equation}\label{equ:KCL7}
 -(R_l^{-1}+R_{i,n}^{-1})V_{i,n}+ R_l^{-1}V_{i,n-1}+R_{i,n}^{-1}V'_{i,n}=0,
\end{equation}
\begin{equation}\label{equ:KCL8}
 -(R_l^{-1}+R_{m,j}^{-1})V'_{m,j}+ R_l^{-1}V'_{m-1,j}+R_{m,j}^{-1}V_{m,j}=0.
\end{equation}

Then, for clarity, a ${2mn\times 1}$ vector ${V}$ is defined to represent
all of the variables in the KCL equations:
\begin{equation}\label{equ:V1}
{V}=[{V_1}^T,{V_2}^T...{V_m}^T,{V'_1}^T,{V'_2}^T...{V'_m}^T]^T,
\end{equation}
where,
%\begin{equation}\label{equ:V2}
%{V_i} = [V_{i,1},V_{i,2}...V_{i,n}]^T,\\
%\end{equation}
%\begin{equation}\label{equ:V3}
%{V'_i} = [V'_{i,1},V'_{i,2}...V'_{i,n}]^T,
%\end{equation}
\begin{equation}\label{equ:V2}
{V_i} = [V_{i,1},V_{i,2}...V_{i,n}]^T,~~{V'_i} = [V'_{i,1},V'_{i,2}...V'_{i,n}]^T,
\end{equation}
for $i=1,2...m$. Then all of the KCL equations can be considered as a
system of linear equations, which has the form
\begin{equation}\label{equ:matrix}
A\cdot V = C.
\end{equation}
$A$ is a ${2mn\times{2mn}}$ coefficient matrix, which is determined by
Equations(\ref{equ:KCL1})-(\ref{equ:KCL8}). $C$ is a ${2mn\times{1}}$
vector, containing the constant terms of these equations. Obviously, the
KCL equation for each point have relatively simple structure and are
similar to each other. Thus, the linear equation system has a fixed format
and simple structure, which is easy to establish and adjust according to
different design schemes and cell parameters. Besides, matrix $A$ is
populated primarily with zeros and can be saved as a sparse matrix, which
will further reduce the storage cost during the computation.

The characteristics of the linear system can be summarized as:
\begin{enumerate}
  \item As shown in Equation~(\ref{equ:blockedmatrix}), the
      coefficient matrix $A$ can be further partitioned into four
      subblocks :
    \begin{equation}\label{equ:blockedmatrix}
        \mathbf{A} = \left[
        \begin{array}{cc}
            A1 & A2  \\
            A3 & A4  \\
        \end{array} \right].
    \end{equation}
All of these subblocks have the same size of $mn\times mn$. Subblock
$A2$ and $A3$ are diagonal matrixes and have the value of: $A2_{i,i} =
A3_{i,i} = R_{i,i}^{-1}$.
\begin{equation}\label{equ:A2A3}
        A2 = A3 = \left[
        \begin{array}{cccc}
            R_{1,1}^{-1}    & 0             & \ldots    & 0 \\
            0               & R_{2,2}^{-1}  & \ddots    & \vdots  \\
            \vdots          & \ddots        & \ddots    & 0 \\
            0               & \ldots        & 0         & R_{mn,mn}^{-1} \\
        \end{array} \right].
    \end{equation}
$A2$ and $A3$ do not change their values with different schemes.

However, $A1$ and $A4$ are a little more complex than $A2$ and $A3$.
$A1$ is a tridiagonal matrix and nonzero elements only locate in the
main diagonal, and the first line below and above the diagonal.
Similarly, $A_4$ is a special tridiagonal matrix, which has nonzero
elements in the main diagonal, and the $n^{th}$ line below and above
the diagonal, where $n$ is the number of bitline in the cross-point
model. The value of the elements in $A1$ and $A4$ can be easily
derived from Equation (\ref{equ:KCL1}) and (\ref{equ:KCL2}). However,
the edge condition varies with different program schemes. Therefore,
the coefficients related to the edge condition should be set according
to the program schemes.


Clearly, the four edges shown in Figure~\ref{fig:modeling} correspond
to different coefficients in $A1$ and $A4$. Due to the space
limitations, we consider the nodes at the left edge of the array as an
example. A similar procedure can be followed to initiate the
coefficients of other edge. The coefficients of nodes at the left edge
of the array ($V_{i,1}$) can be set as:

    \begin{equation}
    A1(k,k) = \left\{
    \begin{array}{ll}
    -(R_l^{-1}+R_{i,1}^{-1})   & \text{if } floating\\
    -(R_v^{-1}+R_l^{-1}+R_{i,1}^{-1})& \text{if } activated
    \end{array} \right.
    \end{equation}
    where $k=(n-1)i+1$ for $i=1,2...m$.

\item The constant terms $C$ is a $2mn{\times}1$ vector.
    Equation(\ref{equ:KCL1})-(\ref{equ:KCL8}) show that only KCL
    equations of the activated points have constant terms. Therefore,
    only the following elements in $C$ may have non-zero value:
    $C((i-1)n+1)$, $C(in)$, $C(mn+i)$ and $C((2m-1)n+i)$ for
    $i=1,2...m$, corresponding to the nodes at the four edges
    respectively. Likewise, as an example, we consider nodes
    $V_{i,1}$. The constant corresponding to these nodes can be
    defined as:
    \begin{equation}
    C((i-1)n+1) = \left\{
    \begin{array}{ll}
    0   & \text{if } floating\\
    -R_v^{-1}V_{Wi}& \text{if } activated
    \end{array} \right.
    \end{equation}
\end{enumerate}
