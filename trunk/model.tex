\vspace{10pt}
\section{Modeling of the Cross-Point Memory}\label{sec:model}

In this section, a detailed mathematical model of cross-point memory array is built. By using the proposed model with specific parameters and
boundary conditions, different read/write schemes can be easily evaluated
at the very early stage of the design.

\subsection{Basic model of Cross-Point Memory}
Figure.~\ref{fig:modeling} shows the circuit model of the cross point ReRAM array. The horizontal lines are word line and vertical lines represent the bit line. The ReRAM cells are located at each cross point of one word line and one bit line. The resistance of the ReRAM cell at the cross point of the $i^{th}$ word line and $j^{th}$ bit line is indicated as $R_{i,j}$. We assume the resistance of the interconnect nanowires between two adjacent cross point has the same value of $R_{line}$. The input resistance of each word line and bit line is $R_v$ and the resistance of sense amplifier is $R_s$. In order to set up the Kirchhoff's Current Law (KCL) equations, the voltage at each cross point is indicated as $V_{i,j}$ for word line and $V'_{i,j}$ for bit line. A detailed cross point is also shown in Figure.~\ref{fig:modeling}. Besides, the input voltage for the $i^{th}$ word line is $V_{W1}$ and for the $i^{th}$ bit line is $V_{B1}$. In the case of two side 

\begin{figure}
\centering
  % Requires \usepackage{graphicx}
  \includegraphics[width=0.47\textwidth]{./figures/model.pdf}\\
  \caption{The basic model of typical cross point array.}\label{fig:modeling}
\end{figure}
\subsection{Edge Conditions for Different Write/Read Schemes}

\subsection{Analysis of the Computing Complexity}
