\begin{abstract}
Since the conventional memory technologies approaching their scaling limit, the non-volatile memory technologies, such as Phase Change RAM~(PCRAM), Magnetoresistive RAM~(STT-RAM) and Resistive Memory~(ReRAM) have attracted considerable attention because their non-volatility, high access speed, low power consumption and good scalability. Among these emerging memory technologies, the ReRAM has shown great potentials a one of the most promising candidates for future universal memory, due to its simple structure, small cell size and potential for 3D stacking. Besides, the unique non-linearity of ReRAM provides the possibility to build a cross-point structure based ReRAM without CMOS access device, with the smallest cell size of $4F^2$. However, the cross-point structure also suffers from its inherent disadvantages and brings in extra design challenges. In this work, the design challenges of cross-point structure based ReRAM are comprehensively analyzed. In addition to the cell-level analysis, different programming schemes are also discussed in detail. A precise mathematical model is built to perform a comprehensive analysis on the issues of reliability, energy consumption and the area overhead. Based on the study, a detailed design methodology is proposed. With the proposed methodology, designers can explore the most energy/area efficient ReRAM design with different design constraints.
\end{abstract}

\vspace{10pt}
\section{Introduction}\label{sec:intro}
The scaling of traditional memory technologies, such as SRAM and DRAM, is approaching its technological and physical limit. In order to effectively follow the Moore's Law~\cite{moore} in the near further, new memory technologies are desired. In the past few years, the non-volatile technologies, including Phase Change RAM~(PCRAM), Magnetoresistive RAM~(STT-RAM) and Resistive Memory~(ReRAM) have been widely accepted as the candidates for next generation memory to meet the need of higher density, faster access time, and lower power consumption. Among all of these emerging memory technologies, ReRAM has many unique characteristics, including simple structure, non-linearity and high resistance ratio, making it be considered as the most promising technology. Researchers have shown that the state-of-art single-level-cell ReRAM can achieve sub-8ns random access time for both read and write operation with resistance ratio larger than 100~\cite{memristor:ISSCC2011_ITRI}. Also, HP labs and Hynix have already announced that they are going to commercialize the memristor-based ReRAM and predicted that ReRAM could eventually replace the traditional memory technologies~\cite{memristor:HpHynix}.

Different from other non-volatile memory technologies, ReRAM can be implemented in a cross-point style structure without any access devices. Generally speaking, in a nano cross-point array, the bistable ReRAM cell is sandwiched by two layers, orthogonal nanowires, without access devices. In this case, the cell size of ReRAM can be further reduced to $4F^2$ per bit. However, the simplicity of the access device free, cross-point structure cell also brings in additional challenges on the peripheral circuit design as well as the memory organization. There are many literatures that analyzed the design challenges of the cross-point ReRAM array~\cite{crossbar_NANO08_Flocke}\cite{crossbar_NANO2002_Ziegler}\cite{crossbar_NANO2003_Ziegler}\cite{crossbar_TED_2010}. Nevertheless, all of these researches focus on the cross-point memory array itself but do not take into account the overhead of peripheral circuitry and different programming methods. Besides, the analysis of area and energy consumption is also lacking. In this work, we carefully analyzed the design challenges of cross-point structure based ReRAM. A precise mathematical model is built to evaluate the reliability, energy consumption and area of different design schemes and various cell parameters. Based on the study, a detailed design framework is proposed. With the proposed methodology, designers can explore the most energy/area efficient ReRAM design with different design constraints and cell parameters at the very beginning of the design stage. On the other hand, the system designers can also leverage the proposed framework to provide valuable feedback to device researchers to adjust their experiments and offer more useful ReRAM cell. We believe that this kind of two-way communications will be very helpful to accelerate speed-to-market of ReRAM memory.

The rest of this paper is organized as follows. In
Section~\ref{sec:preliminary}, the preliminaries of ReRAM technology and cross-point architecture are introduced. Section~\ref{sec:model} discusses the mathematical model built in this paper and detailed the edge conditions for different write and read schemes. Section~\ref{sec:w_and_r} analysis different design constraints of the write and read operation on the cross-point based ReRAM array. Then in Section~\ref{sec:framwork}, the design methodology for the ReRAM array is proposed based on our mathematical model and simulation results. Finally, the conclusion is presented in Section~\ref{sec:conclusion}.