\vspace{10pt}
\section{Preliminaries and Motivations}

This section provides some preliminaries on ReRAM technology and cross-point architecture. Then the limitations of cross-point architecture are demonstrated by a series of simple example, which motivates the work in this paper.

\subsection{Background of ReRAM technology}
Table.~\ref{table:compare} compares the state-of-art non-volatile memory technologies. Obviously, the ReRAM and STT-RAM are the most promising technologies because they have faster access time than PCM and FeRAM with reasonable endurance. However, although the STT-RAM shows the fastest read/write latency among all non-volatile memory technologies, the structure of the memory cell is complex and it has large cell size. On the other hand, the ReRAM has very simple cell structure and can be implemented as a cross point structure, which can work without access devices. The easy structure provides the possibility of high densigy integration and 3-D stackability to ReRAM based memory. Besides, the ReRAM also have much higher ON-OFF resistance ratio than STT-RAM. Therefore, with all of this advantages, ReRAM based memory is a highly competitive
technology compared to all of the emerging non-volatile memory technologies.

\begin{table}[!b]
  \centering
  \scriptsize
    \scriptsize
  \caption{Comparison of Non-Volatile Memory Technologies}\label{table:compare}
  \vspace{-5pt}
%  \begin{tabular}{|cccccp{3.5cm}|}
  \begin{tabular}{|c|cccc|}
    \hline
    % after \\: \hline or \cline{col1-col2} \cline{col3-col4} ...
    \textbf{Metric} & \textbf{STT-RAM} & \textbf{PCM}    & \textbf{FeRAM} & \textbf{ReRAM}
    \\\hline
    \textbf{Cell Size($F^2$)} & $6-20$ & $4-8$ & $15$ & $4$\\\hline
    \textbf{Read Latency(ns)} &  1-10 & 20-50 & 20-80 & 5-50\\\hline
    \textbf{Write Latency(ns)} & 2-20& 150& 100& 5-50\\\hline
    \textbf{Endurance} &  $10^{15}$ & $10^8$ & $10^{12}$ & $10^{8-10}$\\\hline
  \end{tabular}
  \vspace{-10pt}
\end{table}


%Different from the traditional memory technologies, which use the electron stored in the cell to represent the information, the non-volatile memory use the the phase/state/resistance of the memory cell itself to store the data. Therefore, the nonvolatile memory can retain the stored information without pow supply. This kind of non-volatility make it a potential candidate as the alternative memory technology to replace the DRAM even SRAM technologies.
As implied by the name, the ReRAM use its resistance to represent the stored information. The resistance of a ReRAM cell can be switched between high resistance state (HRS) and low resistance state (LRS) by applying an external voltage across the cell. The resistance switching behavior of the metal oxides have been noticed for several years and attracted great research interest recently for the potential application as next generation non-volatile memory technology. A ReRAM memory cell is usually built on a Metal-Insulator-Metal (MIM) structure. The resistance switching behaviors have been observed in many MIM nanodevice with different metal oxide materials. For example, a $TiO_2$ based MIM structure ReRAM was proposed by HP Labs in 2008~\cite{memristor:missing}. The proposed ReRAM is considered as the first experimental realization and a theoretical model of the fourth fundamental circuit elements, which is predicted by Chua~\cite{memristor:chua} about 40 years ago. The memristor-based ReRAM has a very small cell size of $50\times50 nm^2$ with access time less than 50ns. Another $HfO_2$-based bipolar ReRAM is implemented by ITRI this year with as small as 7.2ns access time~\cite{memristor:ISSCC2011_ITRI}.
 
%The memristor based ReRAM built by HP Labs has a two-terminal, two-layer structure. The top and the bottom electrodes are nanowires made by Pt. Two layers of titanium dioxide are sandwiched between these two electrodes in
%a crossbar architecture. By applying an external voltage across the cell, the memristor can switch between two stable states: ON
%state with low resistance and OFF state with high resistance. A
%positive voltage above a specific threshold will switch the device
%into the OFF state (SET operation) and a negative voltage of the
%same magnitude toggles it to its ON state (RESET operation). T

Although there are several different ReRAM proposed by researchers, all of them can be divided into tow classes: the unipolar ReRAM and the bipolar ReRAM. For a unipolar ReRAM cell, the resistance switching behaviors do not depend on the polarity of the voltage input across the cell and only relate to magnitude and latency of the voltage input. However, for a bipolar ReRAM, the voltage polarity for a ON-to-OFF switching (RESET operation) is different from a OFF-to-ON switching (SET operation). A unipolar ReRAM can be easily stacked on top of diodes to built a one diode one resistor (1D1R) ReRAM. However, as mentioned, the SET and the RESET operation have different latency and therefore the performance is mainly determined by the long voltage pulse. Besides, the control of SET, RESET and read operation without any disturbance is another crucial design challenge, especially in the high speed ReRAM design. Therefore, the reported state-of-art high performance ReRAM technologies are dominated by bipolar ReRAM~[? Added reference here].

\subsection{Cross-Point Architecture}
There are two possible memory structures for ReRAM implementation: the traditional one cell one access device structure and the cross-point structure. 

Memristor-based memory has been firstly proposed by Ho~\cite{memristor:pengli}. There are two possible memory organizations
for memristor based memory: \begin{enumerate}
\item \noindent\textbf{Memory Array Structure.} In the memristor based
    memory array, the conventional memory cell is substituted by the
    memristor where the access device remains to be the MOSFET. This is illustrated in Figure~\ref{fig:arch0}(a). In this structure, since each memristor cell has to be accompanied with a MOSFET access device whose size is much larger than the memristor, the memory cell size is mainly dominated by MOSFET access device rather than the actual memristor, and therefore the area efficiency is affected.

  \item \noindent\textbf{Cross Point Structure.}
  The cross point array is a more area-efficient structure for the memristor based ReRAM~\cite{memristor:Cong}. In the cross point array, the only item at each crossing point is the memristor cell. Therefore, the area of the array is significantly reduced since the large MOSFET access part is removed without considering memory peripheral. For the cross point structure, a two-step writing methodology, ERASE-before-RESET, is used to prevent the unintended writing. In read operation two ways are exhibited for preventing a read failure: the first is to supply the same voltage to the unselected row and selected column. In this way, only the data on the select row is read from the selected column. The disadvantage of this method is the voltage drop on the crossing points of the unselected row and the selected column may not be ideal zero because of variations, and this imposes a limitation on the array size. The second way is a two-step write operation. The disturbance current of the partial selected cell on the selected column will be read out beforehand as a background current. Later the total current, comprised of both partial selected cell and full selected cell, will be read out. The state of the selected cell can then be determined by computing the difference between the total current and background current.

   \end{enumerate}

Sneak Path.
\begin{figure}
\centering
  % Requires \usepackage{graphicx}
  \includegraphics[width=0.45\textwidth]{./figures/crossbar_array.pdf}\\
  \caption{A schematic view of typical cross point architecture.}\label{fig:array}
\end{figure}

\subsection{Motivations}
As aforementioned, although the cross-point structure can provide the fabricate simplicity and area efficiency, it also incur lots of design challenges. Following cases show some examples to demonstrate the disadvantages of the cross-point structure, which motivates the work in this paper.
\begin{enumerate}
  \item \textbf{Reliable Write Operation}\\
  In order to

\begin{figure}
\centering
  % Requires \usepackage{graphicx}
  \includegraphics[width=0.4\textwidth]{./figures/example1_large.pdf}\\
  \caption{Case 1: Voltage Drop Along the Word Line during Write Operation.}\label{fig:exampl1}
\end{figure}

  \item \textbf{Read Margin Disturbance}\\
  123
  \item \textbf{Energy Waste Due to Sneak Pass}\\
  123
\end{enumerate}

~\cite{crossbar_NANO08_Nauenheim}~\cite{memristor:analog}~\cite{moore}

