
\section{Conclusion}\label{sec:conclusion}
\vspace{10pt} ReRAM is a promising candidate for next-generation
non-volatile memory technology. The area efficient cross-point structure
is the most attractive memory organization for ReRAM memory design.
However, problems inherent in the cross-point structure, such as the
existence of sneak current and voltage drops along the nanowires introduce
challenges to the design of reliable ReRAM cross-point array. In this
paper, we first establish a mathematical model for cross-point arrays. We
show that the proposed model has a simple structure and is flexible to
evaluate different write/read schemes. By using this model, we study in
detail how reliability affects the array organization, size, energy
consumption, and area overheads in designing cross-point arrays. The
simulation results show that, the multi-bit write operation is more energy
efficient than single-bit write operation and therefore is more suitable
for energy-constrained design. However, from an area-constrained design,
single-bit write operation is better. Also, we point out that both of the
increase of nonlinearity and scaling of write current of the ReRAM cell
can reduce the energy consumption and area overhead significantly, and it
is favorable for large, energy efficient ReRAM design.
