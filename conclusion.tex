\vspace{10pt}
\section{Conclusion}\label{sec:conclusion}

The ReRAM is a promising candidate of the next-generation non-volatile memory technology. The area efficient cross-point structure is the most attractive memory organization for the ReRAM based memory design. However, intrinsic problems of the cross-point structure, such as the existence of sneak current and the voltage drop along the nanowire introduce extra challenges to the design of reliable ReRAM based memory array. In this paper, a mathematical model for the cross-point array is proposed. We show that the propped model has a vary simple structure and is flexible to evaluate different write/read schemes. By using this model, the design constraints, including the array size, energy consumption, and area overheads, are analyzed in details. Based on the results of our study, a detailed design methodology is proposed, which can help designers explore the most energy/area efficient ReRAM design with different design constraints and parameters at the very early design stage.  At the beginning of the computing stage, the reliable array size is obtained by examining the worst case voltage drop and the read margin requirement. Then an iteration is performed to calculate the energy consumption and area overheads for each array size. The result are collected after the computation. If there is any array organization that meets the design constraints provided at   the computation
