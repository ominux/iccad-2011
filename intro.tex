\begin{abstract}
With conventional memory technologies approaching their scaling limit, emerging non-volatile memory technologies have attracted considerable attention because of their non-volatility, high access speed, low power consumption, and good scalability. Among these emerging memory technologies, the Resistive RAM (ReRAM) has shown great potentials as a promising candidates for future universal memory, due to its simple structure, small cell size ($4F^2$), and compatibility for 3D stacking. Besides, the unique non-linearity of ReRAM provides the possibility to build a cross-point structure based high density RAM without CMOS access device. However, the cross-point structure has inherent disadvantages and brings in extra design challenges. In this work, the design challenges of cross-point structure based ReRAM are analyzed. Based on the circuit characteristics of the cross-point array, a precise mathematical model is built to perform a comprehensive analysis of issues of reliability, energy consumption and the area overhead. In addition to the cell-level analysis, different programming schemes are also discussed in detail. Based on the study, a detailed design methodology is proposed to enable designers to perform early-stage search for the most energy/area efficient ReRAM design under design constraints.
\end{abstract}

%\vspace{10pt}
\section{Introduction}\label{sec:intro}
The scaling of traditional memory technologies, such as SRAM and DRAM, is approaching its technological and physical limit. In order to effectively follow the Moore's Law~\cite{moore} in near future, new memory technologies are desired. In the past few years, non-volatile technologies~(NVM), such as Phase Change RAM~(PCRAM), Magnetoresistive RAM~(STT-RAM), and Resistive RAM~(ReRAM) have been widely accepted as candidates for next generation memory technologies to meet the need of higher density, faster access time, and lower power consumption. Among all of these emerging memory technologies, ReRAM has many unique characteristics, including simple structure, non-linearity and high resistance ratio, making itself one of the most promising technologies. Researchers have shown that the state-of-the-art single-level-cell ReRAM can achieve sub-8ns random access time for both read and write operations with a resistance ratio larger than 100~\cite{ReRAM_ISSCC2011_Sheu}. Also, HP labs and Hynix have already announced plans to commercialize the memristor-based ReRAM and predicted that ReRAM could eventually replace the traditional memory technologies~\cite{memristor:HpHynix}.

Different from other non-volatile memory technologies, ReRAM can be implemented in a cross-point style structure without any access devices. Specifically, in a nano cross-point array, each bistable ReRAM cell is sandwiched by two orthogonal nanowires, without access devices. In this case, the cell size of ReRAM can be further reduced to $4F^2$ per bit. However, the simplicity of the access device free, cross-point structure ReRAM also introduces additional challenges to the peripheral circuit design as well as the memory organization. There are several literatures that analyzed the design challenges of the cross-point ReRAM array~\cite{crossbar_NANO2002_Ziegler,crossbar_NANO08_Flocke,crossbar_TED_2010,crossbar_NANO2003_Ziegler}. Nevertheless, all of these researches focus on the cross-point memory array itself but do not take into account the overhead of voltage drivers and different programming methods. Besides, detailed area and energy analysis is also absent. In this work, we addressed the design challenges of cross-point structure based ReRAM. A precise mathematical model is built to evaluate the reliability, energy consumption, and area overheads for different design schemes and various cell parameters. Based on the study, we propose a detailed design methodology which allows for exploring the most energy/area efficient ReRAM design with different design constraints and cell parameters at the very beginning of the design stage. On the other hand, the system designers can also leverage the proposed framework to provide valuable feedback to device researchers who will in turn adjust ReRAM cell design. We believe that this kind of two-way communications will be very helpful to shorten time-to-market of ReRAM memory.

The rest of this paper is organized as follows. In
Section~\ref{sec:preliminary}, the preliminaries of ReRAM technology and cross-point architecture are introduced. Section~\ref{sec:model} discusses the mathematical model we propose for crossbar structure ReRAM and the edge conditions for different write and read schemes. Section~\ref{sec:w_and_r} analyzes different design constraints of write and read operations on the cross-point based ReRAM array. The energy consumption and area overheads are also analyzed in this section. Then in Section~\ref{sec:framwork}, the design methodology for the ReRAM array is proposed based on our mathematical model and simulation results. Finally, the conclusion is presented in Section~\ref{sec:conclusion}. 