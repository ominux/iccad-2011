\begin{abstract}
With conventional memory technologies approaching their scaling limit,
emerging non-volatile memory technologies have attracted considerable
attention because of their non-volatility, high access speed, low power
consumption, and good scalability. Resistive RAM (ReRAM), with its simple
structure, small cell size ($4F^2$), and support for 3D stacking, has been
a leading candidate among emerging technologies. A key advantage of ReRAM
comes from its non-linear nature, which enable us to build a cross-point
RAM array without having a dedicated access transistor in each cell. While
cross-point design is effective in improving memory density, it has
inherent disadvantages which introduce extra design challenges.
%In this work, we analyze these challenges.
Based on the circuit characteristics of the cross-point array, we propose
a mathematical model to perform a comprehensive analysis of issues of
reliability, energy consumption and the area overhead.In addition to the
cell-level analysis, different programming schemes are also discussed in
detail. The proposed model can enable designers to identify the most
energy/area efficient ReRAM organization that meets specific design goals
early in the design stage.

\end{abstract}

\vspace{10pt}
\section{Introduction}\label{sec:intro}
The scaling of traditional memory technologies, such as DRAM and FLASH, is
approaching its physical limit. In the past few years, emerging
non-volatile technologies~(NVM), such as Phase Change RAM~(PCRAM),
Magnetoresistive RAM~(STT-RAM), and Resistive RAM~(ReRAM) have been widely
studied as potential candidates for the next generation memory
technologies to meet the need of higher density, faster access time, and
lower power consumption. Among all of these emerging memory technologies,
ReRAM has many unique characteristics, including simple structure,
non-linearity and high resistance ratio, making itself one of the most
promising technologies. Researchers have shown that the state-of-the-art
single-level-cell ReRAM can achieve sub-8ns random access time for both
read and write operations with a resistance ratio larger than
100~\cite{ReRAM_ISSCC2011_Sheu}. Also, HP labs and Hynix have already
announced plans to commercialize the memristor-based ReRAM and predicted
that ReRAM could eventually replace traditional memory
technologies~\cite{memristor:HpHynix}.

Unlike other non-volatile memory technologies, ReRAM can be implemented in
a cross-point style structure without any access devices. Specifically, in
a nano cross-point array, each bistable ReRAM cell is sandwiched by two
orthogonal nanowires, without access devices. Thus the area occupied by
each cell is literally the area underneath the intersection of wires,
which is $4F^2$ per bit. However, the simplicity of access device free,
cross-point structure introduces challenges to the peripheral circuit
design and memory organization.

While there has been prior studies on cross-point ReRAM
array~\cite{crossbar_NANO2002_Ziegler,crossbar_NANO08_Flocke,crossbar_TED_2010,crossbar_NANO2003_Ziegler},
they do not consider the effect of voltage drivers and programming methods
to cells. In addition, detailed area and energy analysis is also absent.
In this work, we address the design challenges of cross-point structure
based ReRAM. We build an accurate mathematical model to evaluate memory
reliability, energy consumption, and area overhead for different designs
and cell parameters. Based on this study, we propose a detailed design
methodology which allows for exploring the most energy/area efficient
ReRAM design with different design constraints and cell parameters at the
very beginning of the design stage. On the other hand, the system
designers can also leverage the proposed framework to provide valuable
feedback to device researchers who will in turn adjust ReRAM cell design.
We believe that this kind of collaboration will be very helpful to shorten
time-to-market of ReRAM memory.

The rest of this paper is organized as follows. In
Section~\ref{sec:preliminary}, the preliminaries of ReRAM technology and
cross-point architecture are introduced. Section~\ref{sec:model} discusses
the mathematical model we propose for crossbar structure ReRAM and the
edge conditions for different write and read schemes.
Section~\ref{sec:w_and_r} analyzes different design constraints of write
and read operations on the cross-point based ReRAM array. The energy
consumption and area overheads are also analyzed in this section. Then in
Section~\ref{sec:scale}, the effect of non-linearity and write current on
the design constraints are evaluated. Finally, the conclusion is presented
in Section~\ref{sec:conclusion}.
